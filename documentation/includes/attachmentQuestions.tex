\begin{enumerate}
\item Čo to je Selenium? Na aké činnosti sa v praxi používa?\label{q:1}
\item Napíšte plusy a mínusy pre: automatizované a manuálne testovanie.\label{q:2}
\end{enumerate}
\vspace{2em}
\subsection{Odpoveď na otázku č. \ref{q:1}}
\noindent Selenium projekt vyvýja nástroje a knižnice slúžiace na automatizovanie ovládania internetových prehliadačov. V praxi sa využíva na testovanie webových aplikácii. Počas testovaniu Selenium simuluje uživateľské vstupy. Selenium podporuje väčšinu populárnych prehliadačov (Chrome, Firefox, Opera, Safari, Edge), ale aj prehliadače určené pre operačný systém Android či iOS.
\vspace{2em}
\subsection{Odpoveď na otázku č. \ref{q:2}}
\noindent Automatizované testovanie plusy: rýchly a efektývny proces testovania, šetria peniaze pri dlhodobom vývoji softvéru, samo-dokumentácia kódu \newline
Automatizované testovanie mínusy: pomalý vývoj, vyššie náklady pri menších projektoch  \newline
Manuálne testovanie plusy: menšie náklady pri menších projektoch, väčšia šanca nájdenia problému v grafickom rozhraní, flexibilita -- rýchla adaptácia zmene kódu \newline
Manuálne testovanie mínusy: repetitívne a nudné, niektoré testy sú nerealizovateľné -- ťažko vykonateľne (testovanie nízko úrovňového rozhrania, alebo API) \newline